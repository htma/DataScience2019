% Python Programming S1: Introduction
% created on Aug 30, 2018 by mht

\documentclass[14 pt]{beamer}
\usetheme[
	bullet=circle,		% Other option: square
	bigpagenumber,		% circled page number on lower right
	topline=true,			% colored bar at the top of the frame 
	shadow=false,			% Shading for beamer blocks
	]{Flip}

\definecolor{UBCblue}{rgb}{0.04706, 0.13725, 0.26667} % UBC Blue (primary)
\usecolortheme[named=UBCblue]{structure}
\setbeamertemplate{frametitle}{\color{UBCblue}\bfseries\insertframetitle\par\vskip-6pt\hrulefill}

%\usecolortheme[named=Mahogany]{structure} % Sample dvipsnames color

\usepackage{amsmath}
\usepackage{amsfonts}
\usepackage{amssymb}
\usepackage{graphicx}

\graphicspath{{images/}}	% Put all images in this directory. Avoids clutter.


\usetikzlibrary{backgrounds}
\usetikzlibrary{mindmap,trees}	% For mind map

% increase space between items
\let\olditem\item
\renewcommand{\item}{\olditem\vspace{4pt}}

\newcommand{\comment}[1]{\textcolor{comment}{\footnotesize{#1}\normalsize}} % comment mild
\newcommand{\Comment}[1]{\textcolor{Comment}{\footnotesize{#1}\normalsize}} % comment bold
\newcommand{\COMMENT}[1]{\textcolor{COMMENT}{\footnotesize{#1}\normalsize}} % comment crazy bold
\newcommand{\Alert}[1]{\textcolor{Alert}{#1}} % louder alert
\newcommand{\ALERT}[1]{\textcolor{ALERT}{#1}} % loudest alert

\tikzstyle{every picture}+=[remember picture]

\author[mht]{CS}
\title[Deep Learning with PyTorch]{Deep Learning with PyTorch}
\institute{Northeastern University at Qinhuangdao}

\begin{document}
\begin{frame}[c]
\begin{center}
	\textcolor{normal text.fg!50!Comment}{\textbf{\Large{Deep Learning with PyTorch}}}
	\vspace{4em}

    \COMMENT{\large{Lecture 0: Overview}} \\
\vspace{4em}
    \Comment{{Ma Haitao}} \\
\comment{\textit{Northeastern University at Qinhuangdao}}\\
\end{center}
\end{frame}

\begin{frame}{Syllabus}
  \begin{itemize}[<+->]
  \item Module 1 - Introduction to Pytorch
  \item Module 2 - Linear Regression
  \item Module 3 - Classification 
  \item Module 4 - Neural Networks
  \item Module 5 - Deep Networks
  \item Module 6 - Computer Vision Networks
  \end{itemize}
\end{frame}

\begin{frame}{Outline}
  \begin{itemize}
\item What's PyTorch
\item PyTorch compared to Other Deep Learning Frameworks
  \begin{itemize}
  \item TensorFlow
  \item Keras
  \end{itemize}
\item Who's this course for
\end{itemize}
\end{frame}

\begin{frame}{WHAT IS PYTORCH?}

\begin{itemize}
\item A replacement for NumPy to use the power of GPUs
\item A deep learning research platform  
\end{itemize}
\end{frame}
\begin{frame}

\begin{center}
\Large{PyTorch compared to Other \\
Deep learning Frameworks}
\end{center}
\end{frame}

\begin{frame}{Imperative Programming}
  \begin{itemize}
  \item Imperative programming defines computation as you type it
  \item Feels more like Python
  \end{itemize}
  \begin{block}{}
import torch \\
a = torch.tensor(1.0) \\
b = torch.tensor(1.0)\\
c = a + b \\
print('c: ', c)
    \end{block}
c: tensor(2.)
\end{frame}

\begin{frame}
  \frametitle{Symbolic Programming}
  \begin{block}{}
    import tensorflow as tf\\
    a = tf.constant(1.0, name='a')\\
    b = tf.constant(1.0, name='b')\\
    c = a + b\\
    print(c)
  \end{block}  
Tensor(``add:0'', shape=(), dtype=float32)
\begin{block}{}
  sess = tf.Session()\\
ouput = sess.run(c)\\
print('Value of c after running graph:'', output)
\end{block}
2.0
\end{frame}

\begin{frame}
\begin{picture}(0,0)(0,0)
    \put(0, -130)
     { \includegraphics[width=\textwidth]{tf-pytorch.pdf}}
   \end{picture}
 \end{frame}

\begin{frame}
\begin{picture}(0,0)(0,0)
    \put(0, -130)
     { \includegraphics[width=\textwidth]{keras-pytorch.pdf}}
   \end{picture}
 \end{frame}


\begin{frame}
  \frametitle{Imperative Programming}
  \begin{itemize}
  \item Each and every level of computation can be accessed
  \item Regular Python tools are easier to use in PyTorch
  \item Can integrate many standard Programming operations like control
    flow statements
  \item As a result you can quickly gain insight in your mode!
  \end{itemize}
  
\end{frame}

\begin{frame}
  \frametitle{Who should take this course}
  Anyone can build new better neural networks

You should have: 
  \begin{enumerate}
\setbeamertemplate{enumerate items}[circle]

\item Basic knowledge of Calculus and Linear Algegra
\item knowledge of Machine Learning
\item Knowledge of Neural Networks
\item Basic Knowledge of Deep Learning
\end{enumerate}
\end{frame}

\begin{frame}{Roadmap}
  \begin{enumerate}
\setbeamertemplate{enumerate items}[circle]
  \item \Alert{W1: Python, Variables, and Functions}
  \item W2: Strings and Designing Functions
  \item W3: Booleans, Import and if Statements
  \item W4: for Loops and Fancy String Manipulation
  \end{enumerate}
  \begin{block}{Lecture 1: Introduction to Python}
      \begin{itemize}
  \item What a Program is
  \item  Installing Python
  \item Types, Values
  \end{itemize}
  \end{block}
\end{frame}

\end{document}
%%% Local Variables:
%%% mode: latex
%%% TeX-master: t
%%% End:
