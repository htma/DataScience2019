% Python Programming S1: Introduction
% created on Aug 30, 2018 by mht

\documentclass[14 pt]{beamer}
\usetheme[
	bullet=circle,		% Other option: square
	bigpagenumber,		% circled page number on lower right
	topline=true,			% colored bar at the top of the frame 
	shadow=false,			% Shading for beamer blocks
	]{Flip}

\definecolor{UBCblue}{rgb}{0.04706, 0.13725, 0.26667} % UBC Blue (primary)
\usecolortheme[named=UBCblue]{structure}
\setbeamertemplate{frametitle}{\color{UBCblue}\bfseries\insertframetitle\par\vskip-6pt\hrulefill}

%\usecolortheme[named=Mahogany]{structure} % Sample dvipsnames color

\usepackage{amsmath}
\usepackage{amsfonts}
\usepackage{amssymb}
\usepackage{graphicx}

\graphicspath{{images/}}	% Put all images in this directory. Avoids clutter.

\usetikzlibrary{backgrounds}
\usetikzlibrary{mindmap,trees}	% For mind map

% increase space between items
\let\olditem\item
\renewcommand{\item}{\olditem\vspace{4pt}}

\newcommand{\comment}[1]{\textcolor{comment}{\footnotesize{#1}\normalsize}} % comment mild
\newcommand{\Comment}[1]{\textcolor{Comment}{\footnotesize{#1}\normalsize}} % comment bold
\newcommand{\COMMENT}[1]{\textcolor{COMMENT}{\footnotesize{#1}\normalsize}} % comment crazy bold
\newcommand{\Alert}[1]{\textcolor{Alert}{#1}} % louder alert
\newcommand{\ALERT}[1]{\textcolor{ALERT}{#1}} % loudest alert

\tikzstyle{every picture}+=[remember picture]

\author[mht]{CS}
\title[Deep Learning with PyTorch]{Deep Learning with PyTorch}
\institute{Northeastern University at Qinhuangdao}

\begin{document}
\begin{frame}[c]
\begin{center}
	\textcolor{normal text.fg!50!Comment}{\textbf{\Large{Deep Learning with PyTorch}}}
	\vspace{4em}

    \COMMENT{\large{Lecture 1: Overview of Tensors}} \\
\vspace{4em}
    \Comment{{Ma Haitao}} \\
\comment{\textit{Northeastern University at Qinhuangdao}}\\
\end{center}
\end{frame}

\begin{frame}{Outline}
  \begin{itemize}
  \item Tensors 1-Dimensions and 2-Dimensions
  \item Derivatives
  \item Datasets 
\end{itemize}
\end{frame}

\begin{frame}{Neural Network and Tensors}

\end{frame}

\begin{frame}
\begin{center}
\Large{Tensors 1D} 
\end{center}
\end{frame}

\begin{frame}{Objectives}
\begin{itemize}
  \item The Basics
  \item Data types
  \item Indexing and Slicing
  \item Basic Operations
  \end{itemize}
\end{frame}

\begin{frame}
  \Large{The Basics}
\end{frame}

\begin{frame}
  \begin{itemize}
  \item 0-D: 1, 2, 0.2 10
  \item 1-D: tensor is simple an array of numbers
    \begin{enumerate}
    \item A row in a data database
    \item A vector
    \item Time series
    \end{enumerate}
  \end{itemize}
\end{frame}

\begin{frame}
  \frametitle{Data Types}
\begin{table}
\footnotesize{ 
  \begin{tabular}{lll}
\hline
    Data type & dtype & Tensor types \\
\hline
    32-bit floating point & torch.float32 or torch.float &
                                                           torch.FloatTensor
    \\
    64-bit floating point & torch.float64 or torch.double &
                                                           torch.DoubleTensor
    \\
  16-bit floating point & torch.float16 or torch.half &
                                                           torch.HalfTensor
    \\
  
    8-bit integer (unsigned) & torch.uint8  &
                                                           torch.ByteTensor
    \\
    8-bit integer (signed) & torch.int8 &
                                                           torch.CharTensor
    \\
    16-bit integer (signed) & torch.int16 or torch.short &
                                                           torch.ShortTensor
    \\
    32-bit integer (signed) & torch.int32 or torch.int &
                                                           torch.IntTensor
    \\
    64-bit integer (signed) & torch.int64 or torch.long & 
                                                           torch.LongTensor
    \\
\hline  
  \end{tabular}
}
\end{table}
\end{frame}

\begin{frame}
\begin{block}{}
import torch \\
$a$ = torch.tensor([0, 1, 2, 3, 4])\\
$a$.dtype : torch.int64 \\
$a$.type() : torch.LongTensor
\end{block}
\begin{block}{}
$a$ = torch.tensor([0.0, 1.0, 2.0, 3.0, 4.0])\\
$a$.dtype :  torch.float32 \\
$a$.type() : torch.FloatTensor
\end{block}
\begin{block}{}
$a$ = torch.FloatTensor([0, 1, 2, 3, 4])\\
$a$.type() : torch.FloatTensor \\
$a$ :  tensor([0.,1.,2.,3.,4.])
\end{block}
\end{frame}

\begin{frame}
\begin{block}{}
$a$ = torch.tensor([0, 1, 2, 3, 4])\\
$a$.size() : torch.Size(5) \\
$a$.ndimension() : 1
\end{block}
\end{frame}

\begin{frame}
\begin{block}{}
$a$ = torch.tensor([0, 1, 2, 3, 4])\\
$a\_col$ = $a$.view(5, 1) \\
$a\_col$ = $a$.view(-1, 1) 
\end{block}
\end{frame}

\begin{frame}
  \frametitle{Tensor and Numpy }
\begin{block}{}
$np\_array$ = np.array([0.0, 1.0, 2.0, 3.0, 4.0])\\
$torch\_tensor$ = torch.from\_numpy($np\_array$) \\
$back\_to\_np$ = torch\_tensor.numpy()
\end{block}

\begin{block}{}
$this\_tensor$  = torch.tensor([0, 1, 2, 3]) \\
$torch\_to\_list$ = $this\_tensor$.tolist() \\
$torch\_to\_list$ : [0, 1, 2, 3]
\end{block}


\begin{block}{}
$this\_tensor$  = torch.tensor([2, 3, 4, 5]) \\
$this\_tensor$[0] : tensor(2) \\
$this\_tensor$[1] : tensor(3) \\
$this\_tensor$[0].item() : 2 \\
$this\_tensor$[1].item() : 3 
\end{block}
\end{frame}

\begin{frame}
  \frametitle{Indexing and Slicing}
\begin{block}{}
$c$  = torch.tensor([2, 3, 4, 5]) \\
$d$ = c[1:4] \\
$d$ : tensor([3,4,5])
\end{block}
\begin{block}{}
$c$  = torch.tensor([2, 3, 4, 5]) \\
$c$[1:4] = torch.tensor([6 ,7 , 8]) \\
$c$ : tensor([2, 6, 7, 8])
\end{block}
\end{frame}

\begin{frame}
\Large{Basic Operations}
\end{frame}

\begin{frame}
  \frametitle{Vector Addition and Subtraction}
  \begin{block}{}
    $u$ = torch.tensor([1.0, 0.0]) \\
    $v$ = torch.tensor([0.0, 1.0]) \\
    $z$ = $u$ + $v$ \\
    $z$ : tensor([1., 1.])
  \end{block}
\end{frame}

\begin{frame}
  \frametitle{Tensor Multiplication with a Scalar}
  \begin{block}{}
    $y$ = torch.tensor([1, 2]) \\
    $z$ = 2 * $y$ \\
    $z$ : tensor([2, 4])
  \end{block}
\end{frame}

\begin{frame}
  \frametitle{Product of two Tensors}
  \begin{block}{}
    $u$ = torch.tensor([1, 2]) \\
    $v$ = torch.tensor([3, 2])\\
    $z$ = $u$ * $v$ \\
    $z$ : tensor([3, 4])
  \end{block}
\end{frame}

\begin{frame}
  \frametitle{Dot Product of two Tensors}
  \begin{block}{}
    $u$ = torch.tensor([1, 2]) \\
    $v$ = torch.tensor([3, 1])\\
    $z$ = torch.dot($u$, $v$) \\
    $z$ : tensor(5)
  \end{block}
\end{frame}

\begin{frame}
  \frametitle{Adding Constant to a Tensor}
  \begin{block}{}
    $u$ = torch.tensor([1, 2, 3, -1]) \\
    $z$ = $u$  + 1 \\
    $z$ : tensor([2, 3, 4, 0])
  \end{block}
\end{frame}

\begin{frame}
  \frametitle{Functions}
  \begin{block}{}
    $a$ = torch.tensor([1.0, -1, 1, -1]) \\
    $mean\_a$ = $a$.mean() \\
    $mean\_a$ : tensor(0.0)
  \end{block}

  \begin{block}{}
    $b$ = torch.tensor([1, -2, 3, 4, 5]) \\
    $max\_b$ = $b$.max() \\
    $max\_b$ : tensor(5)
  \end{block}

  \begin{block}{np.pi}
    $x$ = torch.tensor([0, np.pi/2, np.pi]) \\
    $y$ = torch.sin(x) \\
    $y$ : tensor([0.0, 1.0, -0.0])
  \end{block}
\end{frame}

\begin{frame}
  \begin{block}{}
    $x$ = torch.linespace(-2, 2, steps=5) \\
    $x$ : tensor([-2., -1., 0., 1., 2.]) \\
  \end{block}
\end{frame}
\begin{frame}
  \frametitle{Plotting Mathematical Functions}
  \begin{block}{}
    import matplotlib.pyplot as plt \\

    $x$ = torch.linspace(0, 2*np.pi, 100) \\
    $y$ = torch.sin($x$) \\
    plt.plot(x.numpy(), y.numpy()) \\
    plt.show()
  \end{block}
\end{frame}

\begin{frame}{Plotting Mathematical Functions}
\begin{picture}(0,0)(0,0)
    \put(30, -100)
     { \includegraphics[width=.8\textwidth]{sin_x.png}}
   \end{picture}

\end{frame}

\begin{frame}{Roadmap}
  \begin{enumerate}
\setbeamertemplate{enumerate items}[circle]
  \item \Alert{W1: Python, Variables, and Functions}
  \item W2: Strings and Designing Functions
  \item W3: Booleans, Import and if Statements
  \item W4: for Loops and Fancy String Manipulation
  \end{enumerate}
  \begin{block}{Lecture 1: Introduction to Python}
      \begin{itemize}
  \item What a Program is
  \item  Installing Python
  \item Types, Values
  \end{itemize}
  \end{block}
\end{frame}

\end{document}
%%% Local Variables:
%%% mode: latex
%%% TeX-master: t
%%% End:
