% Python Programming S2: Introduction
% created on Aug 30, 2019 by mht

\documentclass[14 pt]{beamer}
\usetheme[
	bullet=circle,		% Other option: square
	bigpagenumber,		% circled page number on lower right
	topline=true,			% colored bar at the top of the frame 
	shadow=false,			% Shading for beamer blocks
	]{Flip}

\definecolor{UBCblue}{rgb}{0.04706, 0.13725, 0.26667} % UBC Blue (primary)
\usecolortheme[named=UBCblue]{structure}
\setbeamertemplate{frametitle}{\color{UBCblue}\bfseries\insertframetitle\par\vskip-6pt\hrulefill}

%\usecolortheme[named=Mahogany]{structure} % Sample dvipsnames color

\usepackage{amsmath}
\usepackage{amsfonts}
\usepackage{amssymb}
\usepackage{graphicx}

\graphicspath{{images/}}	% Put all images in this directory. Avoids clutter.

\usetikzlibrary{backgrounds}
\usetikzlibrary{mindmap,trees}	% For mind map

% increase space between items
\let\olditem\item
\renewcommand{\item}{\olditem\vspace{4pt}}

\newcommand{\comment}[1]{\textcolor{comment}{\footnotesize{#1}\normalsize}} % comment mild
\newcommand{\Comment}[1]{\textcolor{Comment}{\footnotesize{#1}\normalsize}} % comment bold
\newcommand{\COMMENT}[1]{\textcolor{COMMENT}{\footnotesize{#1}\normalsize}} % comment crazy bold
\newcommand{\Alert}[1]{\textcolor{Alert}{#1}} % louder alert
\newcommand{\ALERT}[1]{\textcolor{ALERT}{#1}} % loudest alert

\tikzstyle{every picture}+=[remember picture]

\author[mht]{CS}
\title[Deep Learning with PyTorch]{Deep Learning with PyTorch}
\institute{Northeastern University at Qinhuangdao}

\begin{document}
\begin{frame}[c]
\begin{center}
	\textcolor{normal text.fg!50!Comment}{\textbf{\Large{Deep Learning with PyTorch}}}
	\vspace{4em}

    \COMMENT{\large{Lecture 1: PyTorch Basics}} \\
\vspace{4em}
    \Comment{{Ma Haitao}} \\
\comment{\textit{Northeastern University at Qinhuangdao}}\\
\end{center}
\end{frame}

\begin{frame}{Outline}
  \begin{itemize}
  \item Matrics
  \item Tensors
  \item Derivatives
\end{itemize}
\end{frame}

\begin{frame}
\begin{center}
\Large{Matrics: Rectangular Array of Numbers}
\end{center}
\end{frame}

\begin{frame}[fragile]{Creating Matrics}
  \begin{block}{Create a list}
\begin{verbatim}
  arr = [[1,2], [3,4]] 
\end{verbatim}
  \end{block}
  \begin{block}{Create numpy array via list}
\begin{verbatim}
  import numpy as np
  np.array(arr) 
\end{verbatim}
  \end{block}
\end{frame}

\begin{frame}[fragile]
  \frametitle{Create matrices with Default Values}
  \begin{block}{Create 2*2 numpy array of 1's}
\begin{verbatim}
  np.ones((2, 2))
\end{verbatim}
  \end{block}
  \begin{block}{Create 2*2 numpy array of random numbers}
\begin{verbatim}
  np.random.rand(2, 2)
\end{verbatim}
  \end{block}
\end{frame}

\begin{frame}[fragile]
  \frametitle{Seeds for Reproducibility}
  \begin{itemize}
  \item Why do we need seeds?
  \item We need seeds to enable reproduction experimental results.
  \end{itemize}
  \begin{block}{Create seed to enable fixed random numbers }
\begin{verbatim}
  np.random.seed(0)
  np.random.rand(2, 2)
\end{verbatim}
  \end{block}
\end{frame}

\begin{frame}
\begin{center}
\Large{Tensors 1D} 
\end{center}
\end{frame}

\begin{frame}{Objectives}
\begin{itemize}
  \item The Basics
  \item Data types
  \item Indexing and Slicing
  \item Basic Operations
  \end{itemize}
\end{frame}

\begin{frame}
  \Large{The Basics}
\end{frame}
\begin{frame}
  \frametitle{What is a Tensor?}
  \begin{itemize}
  \item Tensor is a data structure representing multi-dimensional array. 
  \item It is similar to a NumPy ndarray. 
  \item It’s size is equivalent to the shape of the NumPy ndarray.
  \end{itemize}
\end{frame}
\begin{frame}{Tensor Basics}
  \begin{itemize}
  \item 0-D: 1, 2, 0.2, 10
  \item 1-D:  simple an array of numbers
 \setbeamertemplate{enumerate items}[circle]
    \begin{enumerate}
    \item A row in a data database
    \item A vector
    \item Time series
    \end{enumerate}
  \item 2-D: a matrix of numbers
  \end{itemize}
\end{frame}

\begin{frame}{An Example of 2D Tensors}
\begin{picture}(0,0)(0,0)
    \put(0, -100)
     { \includegraphics[width=\textwidth]{adog.png}}
   \end{picture}
\end{frame}

\begin{frame}
  \frametitle{Data Types}
\begin{table}
\footnotesize{ 
  \begin{tabular}{lll}
\hline
    Data type & dtype & Tensor types \\
\hline
    32-bit floating point & torch.float32 or torch.float &
                                                           torch.FloatTensor
    \\
    64-bit floating point & torch.float64 or torch.double &
                                                           torch.DoubleTensor
    \\
  16-bit floating point & torch.float16 or torch.half &
                                                           torch.HalfTensor
    \\
  
    8-bit integer (unsigned) & torch.uint8  &
                                                           torch.ByteTensor
    \\
    8-bit integer (signed) & torch.int8 &
                                                           torch.CharTensor
    \\
    16-bit integer (signed) & torch.int16 or torch.short &
                                                           torch.ShortTensor
    \\
    32-bit integer (signed) & torch.int32 or torch.int &
                                                           torch.IntTensor
    \\
    64-bit integer (signed) & torch.int64 or torch.long & 
                                                           torch.LongTensor
    \\
\hline  
  \end{tabular}
}
\end{table}
\end{frame}

\begin{frame}{Create 1-D Tensors}
\begin{block}{}
import torch \\
$a$ = torch.tensor([0, 1, 2, 3, 4])\\
$a$.dtype : torch.int64 \\
$a$.type() : torch.LongTensor
\end{block}
\begin{block}{}
$a$ = torch.tensor([0.0, 1.0, 2.0, 3.0, 4.0])\\
$a$.dtype :  torch.float32 \\
$a$.type() : torch.FloatTensor
\end{block}

\begin{block}{}
$a$ = torch.tensor([0, 1, 2, 3, 4])\\
$a$.size() : torch.Size(5) \\
$a$.ndimension() : 1
\end{block}
\end{frame}

% \begin{frame}
% \begin{block}{}
% $a$ = torch.tensor([0, 1, 2, 3, 4])\\
% $a\_col$ = $a$.view(5, 1) \\
% $a\_col$ = $a$.view(-1, 1) 
% \end{block}
% \end{frame}

\begin{frame}
  \frametitle{ Create 2-D Tensors}
  \begin{block}{Creations}
    a = [[11,12,13], [21,22,23], [31,32,33]]\\
    A = torch.tensor(a) 
  \end{block}

\begin{block}{Built-in Methods}
  A.ndimension() : 2 \\
    A.shape: torch.size([3, 3]) \\
    A.size(): torch.size([3, 3]) \\
    A.numel(): 9 \\
  \end{block}
\end{frame}

\begin{frame}
\begin{center}
\Large{NumPy and Torch Bridge}
\end{center}
\end{frame}

\begin{frame}[fragile]
  \frametitle{NumPy to PyTorch}
  \begin{block}{}
\begin{verbatim}
np_array = np.ones((2, 2))
torch_tensor = torch.from_numpy(np_array)
\end{verbatim}
  \end{block}

  \begin{block}{}
\begin{verbatim}
np.array = np.ones((2,2), dtype=np.int64)
torch_tensor = torch.from_numpy(np_array)
\end{verbatim}
  \end{block}

  \begin{block}{}
\begin{verbatim}
np.array = np.ones((2,2), dtype=np.float32)
torch_tensor = torch.from_numpy(np_array)
\end{verbatim}
  \end{block}
\end{frame}

\begin{frame}[fragile]
  \frametitle{PyTorch to NumPy }
\begin{block}{Create pytorch tensor of 1's}
\begin{verbatim}
  torch_tensor  = torch.ones(2, 2)
  type(torch_tensor)
\end{verbatim}
\end{block}

\begin{block}{Convert tensor to numpy}
\begin{verbatim}
  torch_to_numpy = torch_tensor.numpy()
  type(torch_to_numpy)
\end{verbatim}
\end{block}
\end{frame}

\begin{frame}
  \frametitle{Indexing and Slicing}
\begin{block}{}
$c$  = torch.tensor([2, 3, 4, 5]) \\
$d$ = c[1:4] \\
$d$ : tensor([3,4,5])
\end{block}
\begin{block}{}
$c$  = torch.tensor([2, 3, 4, 5]) \\
$c$[1:4] = torch.tensor([6 ,7 , 8]) \\
$c$ : tensor([2, 6, 7, 8])
\end{block}
\end{frame}

\begin{frame}
  \begin{center}
    \Large{Tensor Operations}
  \end{center}
\end{frame}

\begin{frame}[fragile]
  \frametitle{Basic Operations}
  \begin{block}{Resizing Tensor}
\begin{verbatim}
  a = torch.ones(2, 2)
  a.size(): torch.Size([2, 2])
  a.view(4): tensor([1.,1.,1.,1.])
  a.view(4).size(): torch.Size([4])
\end{verbatim}
  \end{block}
\end{frame}
\begin{frame}[fragile]
  \frametitle{Tensor Additions}
  \begin{block}{Element-wise}
\begin{verbatim}
    X = torch.tensor([[1,0], [0,1]]) 
    Y = torch.tensor([[2,1], [1,2]]) 
    Z = X + Y 
\end{verbatim}
  \end{block}
  \begin{block}{Alternative element-wise addition}
\begin{verbatim}
    Z = torch.add(X, Y)
\end{verbatim}
  \end{block}
  \begin{block}{In-place element-wise addition}
\begin{verbatim}
    X.add_(Y) 
\end{verbatim}
  \end{block}
\end{frame}

\begin{frame}[fragile]
  \frametitle{Adding Constant to a Tensor}
  \begin{block}{}
\begin{verbatim}
    u = torch.tensor([1, 2, 3, -1]) 
    z = u  + 1 
    z : tensor([2, 3, 4, 0])
\end{verbatim}
  \end{block}
\end{frame}

\begin{frame}[fragile]
  \frametitle{Tensor Subtraction}
  \begin{block}{Element-wise}
\begin{verbatim}
    X = torch.tensor([[1,0], [0,1]]) 
    Y = torch.tensor([[2,1], [1,2]]) 
    Z = X - Y 
\end{verbatim}
  \end{block}
  \begin{block}{Alternative element-wise addition}
\begin{verbatim}
    Z = torch.sub(X, Y)
\end{verbatim}
  \end{block}
  \begin{block}{In-place element-wise addition}
\begin{verbatim}
    X.sub_(Y) 
\end{verbatim}
  \end{block}
\end{frame}

\begin{frame}[fragile]
  \frametitle{Tensor Multiplication}
  \begin{block}{Element-wise}
\begin{verbatim}
    X = torch.tensor([[1,0], [0,1]]) 
    Y = torch.tensor([[2,1], [1,2]]) 
    Z = X * Y 
\end{verbatim}
  \end{block}
  \begin{block}{Alternative element-wise addition}
\begin{verbatim}
    Z = torch.mul(X, Y)
\end{verbatim}
  \end{block}
  \begin{block}{In-place element-wise addition}
\begin{verbatim}
    X.mul_(Y) 
\end{verbatim}
  \end{block}
\end{frame}

\begin{frame}[fragile]
  \frametitle{Tensor Multiplication with a Scalar}
  \begin{block}{}
\begin{verbatim}
    y = torch.tensor([1, 2]) 
    z = 2 * y
    z : tensor([2, 4])
\end{verbatim}
  \end{block}
\end{frame}

\begin{frame}[fragile]
  \frametitle{Dot Product of two Tensors}
  \begin{block}{}
\begin{verbatim}
    u = torch.tensor([1, 2]) 
    v = torch.tensor([3, 1])
    z = torch.dot(u, v) 
    z : tensor(5)
\end{verbatim}
  \end{block}
\end{frame}


\begin{frame}[fragile]
  \frametitle{Tensor Division}
  \begin{block}{Element-wise}
\begin{verbatim}
    X = torch.tensor([[1,0], [0,1]]) 
    Y = torch.tensor([[2,1], [1,2]]) 
    Z = X / Y 
\end{verbatim}
  \end{block}
  \begin{block}{Alternative element-wise addition}
\begin{verbatim}
    Z = torch.div(X, Y)
\end{verbatim}
  \end{block}
  \begin{block}{In-place element-wise addition}
\begin{verbatim}
    X.div_(Y) 
\end{verbatim}
  \end{block}
\end{frame}


\begin{frame}[fragile]
  \frametitle{Tensor Mean}
  \begin{block}{Mean}
\begin{verbatim}
X = torch.Tensor([1,2,3,4,5])
X.mean(dim=0)

X = torch.Tensor([[1,2,3,4,5], [4,5,6,7,8])
X.mean(dim=1)
\end{verbatim}
  \end{block}
\end{frame}

\begin{frame}[fragile]
  \frametitle{Tensor Standard Deviation}
  \begin{block}{std}
\begin{verbatim}
X = torch.Tensor([1,2,3,4,5])
X.std(dim=0)
\end{verbatim}
  \end{block}
\end{frame}

\begin{frame}[fragile]
  \frametitle{Torch Arange}
  \begin{block}{arange}
\begin{verbatim}
  X = torch.Tensor(torch.arange(1, 11))
\end{verbatim}
  \end{block}
\end{frame}


\begin{frame}[fragile]
  \frametitle{Other Functions}
  \begin{block}{max}
\begin{verbatim}
    b = torch.tensor([1, -2, 3, 4, 5]) 
    max_b = b.max() 
    max_b : tensor(5)
\end{verbatim}
  \end{block}

  \begin{block}{linspace}
\begin{verbatim}
    x = torch.linspace(-2, 2, steps=5) 
    x : tensor([-2., -1., 0., 1., 2.]) 
\end{verbatim}
  \end{block}
\end{frame}
\begin{frame}[fragile]
  \frametitle{Plotting Mathematical Functions}
  \begin{block}{np.pi}
\begin{verbatim}
    x = torch.tensor([0, np.pi/2, np.pi]) 
    y = torch.sin(x) 
    y : tensor([0.0, 1.0, -0.0])
\end{verbatim}
  \end{block}

  \begin{block}{plot}
\begin{verbatim}
    import matplotlib.pyplot as plt 

    x = torch.linspace(0, 2*np.pi, 100) 
    y = torch.sin(x) 
    plt.plot(x.numpy(), y.numpy()) 
    plt.show()
\end{verbatim}
  \end{block}
\end{frame}

\begin{frame}{Plotting Mathematical Functions}
\begin{picture}(0,0)(0,0)
    \put(30, -100)
     { \includegraphics[width=.8\textwidth]{sin_x.png}}
   \end{picture}
\end{frame}

\begin{frame}
\begin{center}
\Large{Gradients with PyTorch}
\end{center}
\end{frame}

\begin{frame}[fragile]
  \frametitle{Creating Tensors with Gradients}
  \begin{block}{Method 1}
\begin{verbatim}
import torch
a = torch.ones((2, 2), requires_grad=True)

a.requires_grad: True
\end{verbatim}
  \end{block}
  \begin{block}{Method 2}
\begin{verbatim}
a = torch.ones((2, 2))
a.requires_grad_()

a.requires_grad: True
\end{verbatim}
  \end{block}
\end{frame}

\begin{frame}[fragile]
  \frametitle{Creating Tensors without Gradients}
  \begin{block}{}
\begin{verbatim}
import torch
no_gradient = torch.ones((2, 2))

no_gradient.requires_grad: False
\end{verbatim}
  \end{block}
\end{frame}

\begin{frame}[fragile]
  \frametitle{Tensors with Gradient Operations}
  \begin{block}{Addition and Multiplication}
\begin{verbatim}
import torch
a = torch.ones((2, 2), requires_grad=True)
b = torch.ones((2, 2), requires_grad=True)

print(a + b)
print(torch.add(a, b))

print(a * b)
print(torch.mul(a, b))
\end{verbatim}
  \end{block}
\end{frame}

\begin{frame}
  \begin{center}
    \Large{Manually and Automatically Calculating Gradients}
  \end{center}
\end{frame}

\begin{frame}
  \frametitle{What exactly is $requires\_grad$?}
  \begin{itemize}
  \item If you set torch tensor's attribute \emph{requires\_grad} as True,
    it starts to track all operations on it.
  \item when you finish your computation you can call \emph{.backward()}
    and have all the gradients computed automatically.
  \item the gradient for this tensor will be accumulated into \emph{.grad} attribute.
  \end{itemize}
\end{frame}

  \begin{frame}[fragile]{An Example of Linear Equation}
  \begin{displaymath}
    \begin{array}{lll}
          y_i & = &  5(x_i + 1)^2 , \\ 
          z & = & \frac{1}{2}\sum_i y_i
    \end{array}
  \end{displaymath}
    \begin{block}{PyTorch Code}
\begin{verbatim}
  x = torch.ones(2, requires_grad=True)
  y = 5 * (x + 1) ** 2
  z = (1/2) * torch.sum(y)
\end{verbatim}
    \end{block}
  \end{frame}

  \begin{frame}[fragile]
    \frametitle{Calculate Derivatives}
    \begin{block}{Torch backward()}
\begin{verbatim}
  z.backward()
  x.grad

  print(x.requires_grad)
  print(y.requires_grad)
  print(z.requires_grad)
\end{verbatim}
    \end{block}
  \end{frame}

\begin{frame}[fragile]
  \frametitle{More Examples }
  \begin{block}{One tensor}
\begin{verbatim}
  x = torch.tensor(2, requires_grad=True) 
  z = x**2+2*x+1 
  z.backward() 
  x.grad:
\end{verbatim}
  \end{block}
\end{frame}

\begin{frame}[fragile]
  \frametitle{Partial Derivatives}
  \begin{block}{Two tensors}
\begin{verbatim}
  u = torch.tensor(1, requires_grad=True) 
  v = torch.tensor(2, requires_grad=True) 
  f = u*v + u**2
  f.backward() 
  u.grad:  
  v.grad:
\end{verbatim} 
  \end{block}
\end{frame}

\begin{frame}[fragile]
  \frametitle{Accumulate Gradients}
  \begin{block}{One tensor}
\begin{verbatim}
  x = torch.ones(1, requres_grad=True)
  y = x ** 2
  z = x ** 3

  y.backward()
  x.grad

  z.backward()
  x.grad
\end{verbatim}
  \end{block}
\end{frame}

\begin{frame}[fragile]
\footnotesize{
  \frametitle{Cool Hack}
  \begin{block}{Calculate derivates}
\begin{verbatim}
  x = torch.linspace(-10,10,10, requires_grad=True)
  y = x**2 
  z = torch.sum(y)
  z.backward()
\end{verbatim}
  \end{block}
  \begin{block}{Plotting derivates}
\begin{verbatim}
plt.plot(x.detach().numpy(), y.detach().numpy(),
                                     label='function')
plt.plot(x.detach().numpy(), x.grad.detach().numpy(), 
                                   label='derivative')
plt.legend()
plt.show()
\end{verbatim}
  \end{block}
}
\end{frame}

\begin{frame}
    \frametitle{Cool Hack Ploting}
    \begin{figure}
      \centering
      \includegraphics[width=.9\textwidth]{cool_hack.png}
    \end{figure}
\end{frame}

% \begin{frame}
%   \frametitle{ReLU}
%   \begin{block}{}
%     import torch.nn.functional as F\\
%     $x$ = torch.linspace(-3,3,100, requires$\_$grad=True)\\
%     $Y$ = $F.relu(x)$ \\
%     $y = $ torch.sum(F.relu(x))) \\
%     $y$.backward()
%   \end{block}
%   \begin{block}{}
%     plt.plot(x.detach().numpy(), Y.detach(), label='function')\\
% plt.plot(x.detach().numpy(), x.grad.detach().numpy(), label='derivative')\\
% plt.legend()
%   \end{block}
% \end{frame}

\begin{frame}{Roadmap}
  \begin{enumerate}
\setbeamertemplate{enumerate items}[circle]
  \item \Alert{W1: Python, Variables, and Functions}
  \item W2: Strings and Designing Functions
  \item W3: Booleans, Import and if Statements
  \item W4: for Loops and Fancy String Manipulation
  \end{enumerate}
  \begin{block}{Lecture 1: Introduction to Python}
      \begin{itemize}
  \item What a Program is
  \item  Installing Python
  \item Types, Values
  \end{itemize}
  \end{block}
\end{frame}

\end{document}
%%% Local Variables:
%%% mode: latex
%%% TeX-master: t
%%% End:
