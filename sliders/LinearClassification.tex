% Python Programming S2: Linear Classification
% created on Aug 14, 2019

\documentclass[14 pt]{beamer}
\usetheme[
	bullet=circle,		% Other option: square
	bigpagenumber,		% circled page number on lower right
	topline=true,			% colored bar at the top of the frame 
	shadow=false,			% Shading for beamer blocks
	]{Flip}

\definecolor{UBCblue}{rgb}{0.04706, 0.13725, 0.26667} % UBC Blue (primary)
\usecolortheme[named=UBCblue]{structure}
\setbeamertemplate{frametitle}{\color{UBCblue}\bfseries\insertframetitle\par\vskip-6pt\hrulefill}

%\usecolortheme[named=Mahogany]{structure} % Sample dvipsnames color

\usepackage{amsmath}
\usepackage{amsfonts}
\usepackage{amssymb}
\usepackage{graphicx}

\graphicspath{{images/}}	% Put all images in this directory. Avoids clutter.

\usetikzlibrary{backgrounds}
\usetikzlibrary{mindmap,trees}	% For mind map

% increase space between items
\let\olditem\item
\renewcommand{\item}{\olditem\vspace{4pt}}

\newcommand{\comment}[1]{\textcolor{comment}{\footnotesize{#1}\normalsize}} % comment mild
\newcommand{\Comment}[1]{\textcolor{Comment}{\footnotesize{#1}\normalsize}} % comment bold
\newcommand{\COMMENT}[1]{\textcolor{COMMENT}{\footnotesize{#1}\normalsize}} % comment crazy bold
\newcommand{\Alert}[1]{\textcolor{Alert}{#1}} % louder alert
\newcommand{\ALERT}[1]{\textcolor{ALERT}{#1}} % loudest alert

\tikzstyle{every picture}+=[remember picture]

\author[mht]{CS}
\title[Deep Learning with PyTorch]{Deep Learning with PyTorch}
\institute{Northeastern University at Qinhuangdao}

\begin{document}
\begin{frame}[c]
\begin{center}
	\textcolor{normal text.fg!50!Comment}{\textbf{\Large{Deep Learning with PyTorch}}}
	\vspace{4em}

    \COMMENT{\large{Lecture *: Linear Classification}} \\
\vspace{4em}
    \Comment{{Ma Haitao}} \\
\comment{\textit{Northeastern University at Qinhuangdao}}\\
\end{center}
\end{frame}

\begin{frame}{Outline}
  \begin{itemize}
  \item ***
  \end{itemize}
\end{frame}

\begin{frame}{Image Classification}
  \begin{itemize}
  \item the task of assigning an input image one label from a fixed set of
    categories
  \item one of the core problems in Computer Vision that has a large variety of practical applications
  \item  many other seemingly distinct Computer Vision tasks (such as
    object detection, segmentation) can be reduced to it
  \end{itemize}
\end{frame}

\begin{frame}{An Example}
  \begin{picture}(0,0)(0,0)
    \put(30, -100)
     { \includegraphics[width=.8\textwidth]{cat_classify.png}}
   \end{picture}
\end{frame}

\begin{frame}{Challenges for Computers}
  \begin{picture}(0,0)(0,0)
    \put(-5, -50)
     { \includegraphics[width=1\textwidth]{challenges.jpeg}}
   \end{picture}
\end{frame}

\begin{frame}{Data-driven approach}
  \begin{itemize}
  \item  How might we go about writing an algorithm that can classify
    images into distinct categories? 
  \item We provide the computer with many examples of each
    class and then develop learning algorithms that look at these examples
    and learn about the visual appearance of each class.
  \item This approach is referred to as a \emph{data-driven approach},
    since it relies on first accumulating a training dataset of labeled
    images.
  \end{itemize}
\end{frame}

\begin{frame}{An Example Training Set}
  \begin{picture}(0,0)(0,0)
    \put(-5, -80)
     { \includegraphics[width=1\textwidth]{trainset.jpg}}
   \end{picture}
\end{frame}

\begin{frame}{The Image Classification Pipeline}
  % The task in Image Classification is to take an array of pixels that
  % represents a single image and assign a label to it.
\begin{itemize}
\item \textbf{Input}: Our input consists of a set of $N$ images, each labeled
  with one of $K$ different classes. [ \emph{training
  set}].
\item \textbf{Learning}: Our task is to use the training set to learn what
  every one of the classes looks like. [ \emph{training a
  classifier}, or \emph{learning a model}].
\item \textbf{Evaluation}: We evaluate the quality of the
  classifier by asking it to predict labels for a new set of images that it
  has never seen before. We will then compare the true labels of these
  images to the ones predicted by the classifier. 
%Intuitively, we’re
 % hoping that a lot of the predictions match up with the true answers
  %(\emph{ground truth}).
\end{itemize}
\end{frame}

\begin{frame}{CIFAR-10 Dataset}
  \begin{itemize}
  \item This dataset consists of 60,000 tiny images that are 32 pixels high
    and wide
  \item Each image is labeled with one of 10 classes (for example
    “airplane, automobile, bird, etc”). 
  \item These 60,000 images are partitioned into a training set of 50,000
    images and a test set of 10,000 images.
  \end{itemize}
\end{frame}

\begin{frame}{Linear Classification Overview}
Two components:
  \begin{itemize}
  \item  a \textbf{score function} that maps the raw data to class scores, 
  \item and a \textbf{loss function} that quantifies the agreement between the
    predicted scores and the ground truth labels.
  \end{itemize}
  We will then cast this as an optimization problem in which we will
  minimize the loss function with respect to the parameters of the score
  function.
\end{frame}

\begin{frame}{Parameterized mapping from images to label scores}
  \begin{itemize}
  \item training dataset of images:  $x_i \in R^D$, each associated with a
    label $y_i$, Here $i=1\ldots N$ and $y_i \in 1 \ldots K$.
%That is, we have N examples (each
%  with a dimensionality D) and K distinct categories. 
\item In
  CIFAR-10,  we have a training set of $N = 50,000$ images, each with $D = 32 \times
  32 \times 3 = 3072$ pixels, and $K = 10$.
\item  Define the score function $f:R^D \mapsto R^K$ that
  maps the raw image pixels to class scores.
\end{itemize}
\end{frame}

\begin{frame}{Linear Classifier}
  \begin{displaymath}
    f(x_i,W,b)=Wx_i+b
  \end{displaymath}
  \begin{itemize}
  \item the image $x_i$:   a single column vector of shape $[D \times 1]$
  \item  the weights $W$:  a matrix of size $[K \times D]$, 
  \item the bias $b$:   a vector of size $[K \times 1]$
  \item $W$ and $b$ are the parameters of the function
  \end{itemize}
 % In CIFAR-10, $x_i$ is $[3072 \times 1]$ column, $W$ is $[10 \times 3072]$
  %and $b$ is $[10 \times 1]$.
% so 3072 numbers come into the function (the raw pixel values) and 10 numbers come out (the class scores). 
\end{frame}
\begin{frame}{Few Things to Note}
  \begin{itemize}
  \item The single matrix multiplication $Wx_i$ is
    effectively evaluating 10 separate classifiers in parallel (one for
    each class)
  \item the input data $(x_i,y_i)$ as given and
    fixed, but we have control over the setting of the parameters $W,b$
%Our    goal will be to set these in such way that the computed scores match
    %the ground truth labels across the whole training set. 
  \item  once the learning is complete we can
    discard the entire training set and only keep the learned
    parameters
  \item  classifying the test image involves a single
    matrix multiplication and addition, which is significantly faster
  \end{itemize}
\end{frame}

\begin{frame}{Bias Trick- $ f(x_i, W) = Wx_i$}

\begin{picture}(0,0)(0,0)
    \put(-5, -80)
     { \includegraphics[width=1\textwidth]{wb.jpeg}}
   \end{picture}
\end{frame}

\begin{frame}{Loss function}
  \begin{itemize}
  \item Score function defined from the pixel values to class scores, which
    was parameterized by a set of weights $W$. 
  \item we do have control over these weights and we want to set them so that
    the predicted class scores are consistent with the ground truth labels
    in the training data
  \item Loss Function  measures our unhappiness with outcomes (\emph{cost
    function}, or \emph{the objective})
%m  the loss will be high if we’re doing a poor job of classifying the training data, and it will be low if we’re doing well.
\end{itemize}
\end{frame}

\begin{frame}{  Multiclass Support Vector Machine loss}
  \begin{itemize}
  \item The SVM loss is set up so that the SVM ``wants'' the correct class
    for each image to a have a score higher than the incorrect classes by
    some fixed margin $\Delta$. 
  \item The SVM ``wants'' a certain outcome in the sense that the outcome
    would yield a lower loss (which is good).
  \end{itemize}
\end{frame}

\begin{frame}{SVM Loss}
  \begin{itemize}
  \item  $x_i$: the $i$-th example pixels of image 
  \item $y_i$: the label that specifies the index of the correct class
  \item $s$: the class scores computed by the score function takes the pixels and computes the vector
    f(xi,W) 
  \item $s_j$:  the score for the $j$-th class is the j-th element:
    $s_j=f(x_i,W)_j$
  \end{itemize}
\end{frame}

\begin{frame}{}
   The Multiclass SVM loss for the $i$-th example is:
   \begin{displaymath}
     L_i = \sum_{j\neq y_i}max(0, s_j-s_{y_i}+\Delta)
   \end{displaymath}
\end{frame}

\begin{frame}
  \frametitle{Example of SVM Loss}
  \begin{itemize}
  \item $s=[13, -7, 11]$
  \item $y_i=0$
  \item $\Delta=10$
  \end{itemize}
  \begin{displaymath}
    L_i = ?
  \end{displaymath}
  The SVM loss function wants the score of the correct class $y_i$ to be
  larger than the incorrect class scores by at least by $\Delta$. If this
  is not the case, we will accumulate loss.
\end{frame}

\begin{frame}
  \frametitle{SVM Loss for Linear  Functions}
Score Function:
  \begin{displaymath}
    f(x_i; W) = Wx_i
  \end{displaymath}
SVM Loss Function:
  \begin{displaymath}
         L_i = \sum_{j\neq y_i}max(0, w_j^Tx_i-w_{y_i}^Tx_i+\Delta)
  \end{displaymath}
where $w_j$ is the $j$-th row of $W$ reshaped as a column.

the threshold at zero $max(0,-)$ function is often called the \emph{hinge
  loss}.
\end{frame}
\begin{frame}
  \begin{centering}
    \emph{The loss function quantifies our unhappiness with predictions on
      the training set.}
  \end{centering}
 \end{frame}

 \begin{frame}
   \frametitle{Regularization}
   \begin{itemize}
\item We wish to encode some preference for a certain set of weights $W$
\item We can extend the loss function with a regularization penalty $R(W)$
\item The most common regularization penalty is the \textbf{L2} norm that discourages large weights through an elementwise quadratic penalty over all parameters:
  \begin{displaymath}
    R(W) = \sum_k\sum_lw_{k,l}^2
  \end{displaymath}
   \end{itemize}
 \end{frame}

 \begin{frame}[fragile]
   \frametitle{ The Full Multiclass SVM Loss }
   The full loss for the dataset is the mean of $L_i$ over all training
   examples together with a regularization term $R(W)$.
   \begin{displaymath}
     L = \underbrace{\frac{1}{N}\sum_i L_i}_{data \ loss} +
     \underbrace{\lambda R(W)}_{regularization \ loss}
   \end{displaymath}
\small{
   \begin{displaymath}
     L   =   \frac{1}{N}\sum_i\sum_{j\neq y_i}[max(0, f(x_i;W)_j - f(x_i;
     W)_{y_j}+\Delta)]   +      \lambda \sum_i\sum_lW_{k,l}^2
   \end{displaymath}
}
Where $N$ is the number of training examples, $\lambda$ is a hyperparameter.
 \end{frame}

  \begin{frame}[fragile]
    \frametitle{Python Implementation}
\small{
    \begin{block}{}
 \begin{verbatim}
def L_i_vectorized(x, y, W):
  delta = 1.0
  scores = W.dot(x)
  margins = np.maximum(0, scores-scores[y]+delta)
  margins[y] = 0
  loss_i = np.sum(margins)
  return loss_i
 \end{verbatim}
    \end{block}
}
  \end{frame}

\begin{frame}
\begin{center}
\Large{Softmax Classification}
\end{center}
\end{frame}

\begin{frame}
  \frametitle{Softmax Classification}
  \begin{itemize}
  \item the generalization of binary Logistic
   Regression classifier to multiple classes
 \item  the SVM  treats the outputs $f(x_i,W)$ as  scores for each class
 \item  the Softmax classifier gives  scores as the unnormalized log
   probabilities for each class
 \item  replace the hinge loss with a cross-entropy loss
 \end{itemize}
\end{frame}

\begin{frame}
  \frametitle{Cross-Entropy Loss}
  \begin{displaymath}
    L_i = -log(\frac{e^{f_{y_i}}}{\sum_je^{f_j}}) \ or \ equivalently
  \end{displaymath}
  \begin{displaymath}
    L_i = -f_{y_i}+ log\sum_je^{f_j}
  \end{displaymath}
where we are using the notation $f_j$ to mean the $j$-th element of the vector of class scores $f$.
\end{frame}

\begin{frame}
  \frametitle{Softmax Function}
  \begin{displaymath}
  f_j(z) = \frac{e^{z_j}}{\sum_k e^{z_k}}
\end{displaymath}
  \begin{itemize}
  \item softmax function  takes a vector of arbitrary real-valued scores
    (in z) 
  \item squashes it to a vector of values between zero and one that sum
    to one
  \end{itemize}
\end{frame}

\begin{frame}
  \frametitle{SVM vs. Softmax }
\begin{picture}(0,0)(0,0)
    \put(-5, -80)
     { \includegraphics[width=1\textwidth]{svmvssoftmax.png}}
   \end{picture}
\end{frame}

\begin{frame}{Summary}
  \begin{itemize}
  \item We defined a score function $f(x, W, b) = Wx+b$ from image pixels to class scores.
  \item  This parametric approach is that once we learn the parameters we
    can discard the training data and , the prediction for a new test image
    is fast.
  \item We defined a loss function  that measures how compatible a given
    set of parameters is with respect to the ground truth labels in the
    training dataset. 
  \item We also saw that the loss function was defined in such way that
    making good predictions on the training data is equivalent to having a
    small loss.
  \end{itemize}
\end{frame}
\begin{frame}
\begin{center}
  \Large{But how do we efficiently determine the parameters that give the best (lowest) loss? }
\end{center}
\end{frame}

\begin{frame}{Optimization}
  \begin{itemize}
  \item   A (parameterized) score function mapping the raw image pixels to
    class scores 
  \item A loss function that measured the quality of a particular set of
    parameters based on how well the induced scores agreed with the ground
    truth labels in the training data. 
  \item \textbf{Optimization} is the process of finding the set of parameters W that
    minimize the loss function.
  \end{itemize}
  
\end{frame}
\end{document}
%%% Local Variables:
%%% mode: latex
%%% TeX-master: t
%%% End:
